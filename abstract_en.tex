\chapter*{Abstract}
\begin{spacing}{1.0}
In Bachelor's degrees in Computer science or Computer engineering, students are expected to have contact with an introductory-level Compilers course. Due to the large breadth and depth of topics in this subject, professors and instructors tend to conduct it from a more theoretical point of view. 

This approach tends to provide students a deeper understanding of the fundamental concepts of this field and better prepare them better for a research oriented career. However, it tends to limit the hands-on experience of the taught concepts, making students often feel disconnected from real-world applications of this area, causing less engagement, excitement and motivation among them. Given this, since the 90s, new methodologies, which combined theory and practice, began to emerge. In most of them, students we required implement a compiler or an interpreter for a programming language for teaching purposes, normally defined by the responsible professor or instructor.

Due to the aforementioned disadvantages of the traditional approach and the limitations that some of the proposed programming languages in the mixed approach present, this proposal aims to present a new programming language called Bleach along with an implementation of its interpreter. This language is intended to be used as a tool in introductory-level Compilers courses whose responsible professors opt to follow a teaching approach more focused on incremental practice, prioritizing flexibility and aiming to solve, or at least mitigate, the disadvantages of both mentioned approaches.



\textbf{Keywords:} Compilers; C++; Education; Educational Software; Interpreters; Programming Languages.  

\end{spacing}