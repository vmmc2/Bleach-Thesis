\thispagestyle{empty}
\begin{spacing}{1.0}
\chapter*{Resumo}
Nos cursos de bacharelado em Ciência da Computação ou Engenharia da Computação, espera-se que os alunos tenham contato com uma disciplina de Compiladores de nível introdutório. Devido à grande amplitude e complexidade dos temas abordados nesta disciplina, os professores responsáveis tendem a conduzi-lá sob um ponto de vista mais teórico. 

Esta abordagem costuma proporcionar aos alunos uma compreensão mais profunda dos conceitos fundamentais desta área e prepará-los melhor para uma carreira orientada para a pesquisa científica. No entanto, tende a limitar a experiência prática dos conceitos ensinados, fazendo com que os alunos muitas vezes se sintam desconectados das aplicações dessa área, causando um menor engajamento e motivação entre estes. Diante disso, desde a década de 90, novas metodologias, que mesclam teoria e prática, começaram a surgir. Na maioria delas, os alunos devem implementar um compilador ou um interpretador para uma linguagem de programação de fins didáticos, normalmente definida pelo professor responsável.

Devido às desvantagens mencionadas da abordagem tradicional e às limitações que algumas das linguagens propostas na abordagem mista apresentam, a proposta atual visa apresentar uma nova linguagem de programação chamada Bleach, juntamente com uma implementação de seu interpretador. Tal linguagem pretende ser utilizada como ferramenta em disciplinas introdutórias de Compiladores que optem por seguir uma abordagem de ensino mais voltada para a prática incremental, priorizando a flexibilidade e objetivando solucionar, ou pelo menos mitigar, as desvantagens de ambas as abordagens mencionadas. No entanto, é essencial esclarecer que a avaliação de Bleach no contexto educacional não faz parte do escopo do trabalho aqui apresentado.

\textbf{Palavras-chave:} Compiladores; C++; Educação; Software Educacional; Interpretadores; Linguagens de Programação.
        
\clearpage
\end{spacing}