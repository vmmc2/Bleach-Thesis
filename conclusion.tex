\chapter{Conclusion and Future Work} \label{cap:conclusao}

\begin{displayquote}
    \begin{center}
        \textit{``Even without a form, we will never stop walking.''}
    \end{center}
\end{displayquote}

\begin{flushright}
   \textit{-- TITE KUBO}
\end{flushright}

\section{Conclusion}

\section{Future Work}
-> Permitir que o tipo \texttt{str} possa ser passado como referência para funções. \textbf{Refletir se essa escolha faz sentido...}

-> Permitir a declaração de constantes usando a palavra-chave \texttt{const} no lugar da palavra-chave \texttt{let}.

-> Implementar um REPL parecido com o de Python, que permite quebra de linha ao declarar funções, criar if-statements, for loops, etc.

-> Fazer uma passagem de Resolver mais complexa a fim de reportar uma maior variedade e quantidade de erros estáticos.

-> Não consegui fazer indexing para os tipos 'list' e 'str'.

-> Adicionar métodos estáticos a uma classe declarada.

-> Aumentar a quantidade de testes presentes na suite de testes (atualmente são 42).

-> Implementar um for do tipo \texttt{for ... in ... list}.

-> Realizar uma pesquisa de opinião com professores, instrutores e monitores a respeito da opinião deles sobre Bleach. É possível utilizá-la em um ambiente de sala de aula?

