\chapter{Theoretical Review} \label{Cap:theoretical_review}

\begin{displayquote}
    \begin{center}
        \textit{``Do not live bowing down. Die standing up.''}
    \end{center}
\end{displayquote}

\begin{flushright}
   \textit{-- TITE KUBO}
\end{flushright}

\section{Introduction}
This chapter is dedicated to provide a literature review about the Programming Language Design and Compiler/Interpreter Implementation fields. The conceptual content presented here is heavily influenced by \cite{aho1986compilers}, \cite{cooper2022engineering} and \cite{nystrom2021crafting}.


\section{Compilers and Interpreters}

\section{Steps to Implement a Programming Language}

\subsection{Lexing}
The Lexing phase of a compiling/interpreting process, also know as Lexical Analysis, 

\subsection{Parsing}
The Parsing phase of a compiling/interpreting process, also known as Syntax Analysis,

\subsection{Static Analysis}
The Static Analysis phase of a compiling/interpreting process

\subsection{Intermediate Representations}
After the previous phases have been executed

\subsection{Optimization}
The Optimization phase of a compiling/interpreting process

\subsection{Code Generation}
The Code Generation phase of a compiling/interpreting process


\subsection{Virtual Machines}
Virtual Machines are 

\subsection{Runtimes}
Runtimes are

\section{Shortcuts and Alternate Routes}
This section is dedicated to present an overview of a few ad-hoc strategies that might be used when implementing a programming language.

\subsection{Single-Pass Compilers}
A Single-Pass Compiler
\subsection{Tree-Walk Interpreters}
A Tree-Walk Interpreter

\subsection{Transpilers}
A Transpiler

\subsection{Ahead-of-Time Compilers}
An Ahead-of-Time Compiler, also known as AOT Compiler, is a type of compiler

