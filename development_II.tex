\chapter{Bleach} \label{cap:metodologia}

\begin{displayquote}
    \begin{center}
        \textit{``Bullet, Claw, Battle Flag, Short Sword. With my fingers bent, I wait for you.''}
    \end{center}
\end{displayquote}

\begin{flushright}
   \textit{-- TITE KUBO}
\end{flushright}

\section{Introduction}
This chapter is dedicated to provide a detailed explanation about the Bleach programming language. Firstly, a brief overview of Bleach's features will be presented by showcasing a few programs written in the language. Then, a detailed breakdown of each of its components and features will be shown to the reader. Afterwards, the challenges encountered during the development of this project will be presented, as well as decisions and trade-offs made. Finally, a comparison between Bleach and the previous languages introduced will be made, as well as a discussion about how Bleach can be properly used to its fullest in an undergraduate Compilers course.

\section{Bleach Overview}
As previously mentioned, this section is dedicated to showcase a few of Bleach's features through the exposition of simple, yet useful, programs to the reader. It's expected that the reader has some familiarity programming with another languages, such as C \cite{kernighan1988c}, C++ \cite{strousrup2000c++}, JavaScript, Lox \cite{nystrom2021crafting}, Python \cite{python_language} and Ruby, so the similarities between Bleach and these languages make themselves more evident. \newline

\begin{figure}
    \centering
    \begin{lstlisting}
function greet(){
  print "Hello, World!";
}

greet(); // "Hello, World!"
    \end{lstlisting}
    \caption{Bleach's simplest program: The famous "Hello, World!" program.}
\end{figure}

\begin{figure}
    \centering
    \begin{lstlisting}
function factorial(n){
  if(n == 0){
    return 1;
  }else{
    return n * factorial(n - 1);
  }
}

print factorial(5); // 120
    \end{lstlisting}
    \caption{A Bleach program that shows the usage of recursive functions, if-else statements and arithmetic operators to calculate the factorial of a number.}
\end{figure}

\begin{figure}
    \centering
    \begin{lstlisting}
function main(){
  let numbersInfo = [];
  for(let i = 0; i <= 10; i = i + 1){
    if(i % 2 == 0){
      numbersInfo.append(i + " is even!");
    }else{
      numbersInfo.append(i + " is odd!");
    }
  }
  print numbersInfo;
}

main();
    \end{lstlisting}
    \caption{A Bleach program that shows the usage of for loops and the native \texttt{list} type in order to build a list that contains some information about numbers from 0 to 10.}
\end{figure}

\begin{figure}
    \centering
    \begin{lstlisting}
function quadraticEquationSolver(a, b, c){
  let delta = std::math::pow(b, 2) - 4 * a * c;
  let x1 = (-b + std::math::sqrt(delta))/(2 * a);
  let x2 = (-b - std::math::sqrt(delta))/(2 * a);

  return [x1, x2];
}

print quadraticEquationSolver(1, 0, -4); // [2, -2]
    \end{lstlisting}
    \caption{A Bleach that uses a function which receives the coefficients of a quadratic equation to compute its roots.}
\end{figure}

\begin{figure}
    \centering
    \begin{lstlisting}
class Shape{ // Base Class
  method init(name){
    self.name = name;
  }

  method str() {
    return "This is a " + self.name + ".";
  }

  method area(){ // To be overridden by subclasses
    return 0;
  }
}

class Circle inherits Shape{ // Derived Class
  method init(radius){
    super.init("Circle"); // Call the Base Class constructor
    self.radius = radius;
  }

  method area(){
    return std::math::pow(self.radius, 2) * 3.14159;
  }
}

class Triangle inherits Shape{ // Derived Class
  method init(base, height) {
    super.init("Triangle"); // Call the Base Class constructor
    self.base = base;
    self.height = height;
  }

  method area(){
    return (self.base * self.height) / 2;
  }
}

let c = Circle(5);
let t = Triangle(3, 7);

// This is a Circle. It has an area of 78.539749999999998 units.
print c + " It has an area of " + c.area() + " units.";
// This is a Triangle. It has an area of 10.5 units.
print t + " It has an area of " + t.area() + " units.";
    \end{lstlisting}
    \caption{A Bleach program that illustrates its Object-Oriented features, such as: classes, instances, attributes, methods, overriding and inheritance.}
\end{figure}

\begin{figure}
    \centering
    \begin{lstlisting}
function writeToFile(){
  let userInput = std::io::readLine();
  std::io::fileWrite("output.txt", "w", userInput, true);

  userInput = std::io::readLine();
  std::io::fileWrite("output.txt", "a", userInput, false);

  return;
}

writeToFile();
    \end{lstlisting}
    \caption{A Bleach program that shows its capability with dealing with user-input and writing to \texttt{.txt} files.}
\end{figure}

\section{Bleach Features}
\subsection{Introduction}
This section, as its name suggests, is dedicated to provide to the reader a more extensive and exhausting walkthrough the features available in Bleach. 

\subsection{Data Types}
Bleach has 5 built-in data types made available to the user. Such types are divided into scalar types  and compound types.

Scalar types are types that can represent a single value. In Bleach, there are 3 scalar data types: \texttt{bool}, \texttt{nil} and \texttt{num}.
\begin{itemize}
    \item \texttt{bool}: This data type is used to represent one of two possible values: \texttt{true} or \texttt{false}. Such values are used to express logical conditions and to control the execution of certain parts of a program's source code.
    
    In this context, it's important to mention that Bleach takes inspiration from Ruby and other modern programming languages and implements the concept of "truthy" and "falsey" values in order to evaluate the truthiness or the falseness of values when they are being evaluated inside \texttt{if} statements, loops (\texttt{while}, \texttt{do-while}, \texttt{for}) and ternary operators (\texttt{? :}).

    Essentially, this means that, in Bleach, values of any type (built-in or user-defined) can be used in places where a value of type \texttt{bool} is expected. Moreover, Bleach opted to follow the same convention of Ruby in this matter, which says:

    \begin{itemize}
        \item \textbf{"Falsey" values:} \texttt{false}, \texttt{nil}.
        \item \textbf{"Truthy" values:} Any other value that is not \texttt{false} nor \texttt{nil}.
    \end{itemize}
    
    \item \texttt{nil}: This one is an old acquaintance for most programmers. The \texttt{nil} type has just one possible value: the \texttt{nil} value. In short this value conveys the idea of "no value" or the idea of "absence of a value".

    In other programming languages, this concept appears in other forms, like: \texttt{NULL}, \texttt{nullptr}, \texttt{null}, \texttt{None}, \texttt{nil}, and others. Since Bleach has some influence from Ruby, it adopted \texttt{nil}.
    
    \item \texttt{num}: In favor of simplicity, Bleach has only one type to represent numbers: the \texttt{num} type. This type can be used to represent both integers and floating-point numbers.

    Behind the scenes, the \texttt{num} type is implemented with a double-precision floating-point type, which allows Bleach to cover a lot of territory when it comes to numerical values while still keeping the simplicity initially envisioned for the language.
    
    Finally, with respect to numerical literals, Bleach has support for only basic integer and decimal literals, such as: \texttt{23}, \texttt{42}, \texttt{3.14159}, \texttt{2.71828}, \texttt{-10.4}.
    
\end{itemize}

Compound types are types that can group multiple values into one. In Bleach, there are 2 compound types: \texttt{list} and \texttt{str}.
\begin{itemize}
    \item \texttt{list}: Taking inspiration from Python \cite{python_language}, Bleach has this type. Here, the \texttt{list} type represents an linear-sequence of elements. As previously stated, Bleach is a dynamically-typed language, just like Python. Thus, the \texttt{list} type can store values of different types with no issues. Moreover, Bleach has support for \texttt{list} literals, just as the ones shown below:\newline

    \begin{figure}[H]
        \centering
        \begin{lstlisting}
let l0 = [0, 1, 2, 3, 2.71, 3.14159];
let l1 = ["hello", "there"];
let l2 = [];
let l3 = [false, "Brazil", 9.98, nil];
        \end{lstlisting}
        \caption{Examples of \texttt{list} literals in Bleach.}
    \end{figure}

    Finally, as in Python, the \texttt{list} type in Bleach has the following useful methods that makes the student's life easier when working with this type.
    
    \begin{itemize}
        \item \texttt{append}: Method responsible for adding one value of any type to the end of the \texttt{list} value it was called on. Returns \texttt{nil}.
        \item \texttt{clear}: Method responsible for deleting every element that is currently stored inside the \texttt{list} value it was called on. Returns \texttt{nil}.
        \item \texttt{empty}: Method responsible for checking whether the \texttt{list} value it was called on currently has values stored inside it or not. Returns a \texttt{bool} value.
        \item \texttt{fill}: Method responsible for resizing the \texttt{list} value it was called on to a provided size and also filling all of its indexes with a provided value of any type. Returns \texttt{nil}; 
        \item \texttt{getAt}: Method responsible for returning the value present at the provided index. Returns a value of any type.
        \item \texttt{pop}: Method responsible for deleting and returning the last element present inside a \texttt{list} value, if any. Returns a value of any type.
        \item \texttt{setAt}: Method responsible for setting the value stored at the provided index to the value that was provided. Returns \texttt{nil}.
        \item \texttt{size}: Method responsible for checking and returning the current amount of elements that the \texttt{list} value it was on called on currently has. Return a \texttt{num} value.
    \end{itemize}

    Last but not least, it's important to mention that any misuse of the methods presented above will result in a runtime error during the program's execution.

    \item \texttt{str}: Also taking inspiration from Python \cite{python_language}, Bleach has this type. The \texttt{str} type represents an indexed-sequence of characters (a string), typically used to store and manipulate text. Just as the \texttt{list} type, Bleach also has support for literal values of this type:

    \begin{figure}[H]
        \centering
        \begin{lstlisting}
let s0 = "Ichigo Kurosaki";
let s1 = "Byakuya Kuchiki";
let s2 = "Bazzard Black";
let s3 = "Jugram Haschwalth";
        \end{lstlisting}
        \caption{Examples of \texttt{str} literals in Bleach.}
    \end{figure}

    There are other aspects about the \texttt{str} type that must be mentioned.

    First of all, it is important to recall that it's a sequence type. In short, it means that values of this type can be indexed. In a string, indexing usually allows the programmer to access individual characters from the value, which, in the Bleach programming language, are also values of \texttt{str} type.

    Also, as seen in Figure 4.8, in Bleach, literals values of this type are always enclosed by double quotes (\texttt{""}).

    Finally, as in Python, this type has the following useful methods available:

    \begin{itemize}
        \item \texttt{empty}: Method responsible for checking whether the \texttt{str} value it was called is equal to \texttt{""} or not. Returns a \texttt{bool} value.
        \item \texttt{find}: Method responsible for trying to figure out if the provided sub-string (a \texttt{str} value) exists within the \texttt{str} value the method was called on. Returns a \texttt{num} value which denotes the index at which the sub-string appears, otherwise returns \texttt{-1}. 
        \item \texttt{length}: Method responsible for checking and returning the current amount of characters (which are also values of type \texttt{str}) that the \texttt{str} value it was on called on currently has. Returns a \texttt{num} value.
        \item \texttt{split}: Method responsible for generating a \texttt{list} where each value is of \texttt{str} type. This method receives as its unique argument a value of \texttt{str} type that works as the separator. Returns a \texttt{list} value.
        \item \texttt{substr}: Method responsible for retrieving a sub-string from the \texttt{str} value it was called on. The method receives two arguments of \texttt{num} type that work as the start and end delimiters. Returns a \texttt{str} value.
    \end{itemize}

\end{itemize}

\subsection{Comments}
Even though it is recommendable that a programmer writes code in a way that is readable and understandable, there are certain scenarios where extra explanation about what certain parts of a program's source code is actually doing is necessary.

In these specific cases, programmers usually leave comments in their code that the compiler/interpreter will ignore but people reading their code may find helpful.

Given this, Bleach takes inspiration from C \cite{kernighan1988c} when it comes to support for single-line comments and multi-line comments.

\begin{itemize}
    \item \textbf{Single-Line Comments:} In Bleach, a single-line comment starts with two slashes (\texttt{//}), and the comment continues until the end of the line.
    \item \textbf{Multi-Line Comments:} Bleach also has support for multi-line comments. A multi-line comment has a beginning and an ending. The former is denoted by a (\texttt{/*}), while the later is denoted by a (\texttt{*/}). Everything written in-between is considered a comment and, thus, is ignored by the interpreter at runtime.
\end{itemize}

    \begin{figure}[H]
        \centering
        \begin{lstlisting}
// This is a single-line comment.

/*
This 
is
a
multi-line
comment.
*/
        \end{lstlisting}
        \caption{Examples of \texttt{str} literals in Bleach.}
    \end{figure}

\subsection{Variables}
A variable, as in most programming languages, can be viewed as just a name associated with a storage location in memory. Such storage location is responsible for holding data that can be changed during the execution of a program.

In Bleach this is not different. Variables still have this main functionality. However, they have some particularities that might differ from variables in other programming languages. Such peculiarities are listed below:
\begin{itemize}
    \item In Bleach, all variables are mutable. In other words, once a variable is declared, any amount of assignments are allowed to be performed on this declared variable.
    
    \item A direct consequence of the particularity explained above is that there is no concept of constants in Bleach. There are just mutable variables, as previously mentioned.
    
    \item In Bleach, to declare a new variable, the \texttt{let} keyword must be used:
    \begin{figure}[H]
        \centering
        \begin{lstlisting}
let s = "A value of type 'str'";
        \end{lstlisting}
        \caption{Example of variable declaration in Bleach.}
    \end{figure}
    
    \item In Bleach, if the declaration of a variable does not have an initializer, then, by default, the variable will store the nil value:
    \begin{figure}[H]
        \centering
        \begin{lstlisting}
let someVariable;
print someVariable; // nil
        \end{lstlisting}
        \caption{Example of variable declaration without an initializer in Bleach.}
    \end{figure}

    \item Since Bleach is a dynamically-typed programming language, variables don't have types associated with them. Instead, it's the values stored inside them that have types. The most important consequence of this fact is that a variable can hold a value of different types at different points in time:
    \begin{figure}[H]
        \centering
        \begin{lstlisting}
let a = "hello";
print a; // "hello"

a = nil;
print a; // nil

a = 3.14 + 2.71;
print a // 5.85
        \end{lstlisting}
        \caption{Example demonstrating the fact that, in Bleach, variables do not have types associated.}
    \end{figure}
\end{itemize}

Bleach also has support for global variables, those that have been declared outside of all functions, methods or classes, making them accessible from any part of the program's source code. One point that makes Bleach kind of unique is the fact that the programmer is allowed to re-declare a global variable anytime:
\begin{figure}[H]
    \centering
    \begin{lstlisting}
let pi = 3.14159;
// Write some code in the global scope.
let pi = 9.51413;
    \end{lstlisting}
    \caption{Example demonstrating re-declaration of global variables in Bleach.}
\end{figure}

As expected from most programming language, Bleach supports local variables, which are those declared within a specific block of code, such as a function, method, an \texttt{if-elif-else} statement or a loop statement (\texttt{while}, \texttt{do-while}, \texttt{for}). In this regard, it's important to remember the reader that the scope of a local variable is limited to the block in which it is defined, meaning it can only be accessed and used within that block. Furthermore, once the block of code finishes executing, the local variable is normally destroyed, and its memory is released.
\begin{figure}[H]
    \centering
    \begin{lstlisting}
function foo(){
  let bar = "This is a local variable";
  print bar; // "This is a local variable"

  return;
}

foo();
    \end{lstlisting}
    \caption{Example demonstrating the usage of local variables in Bleach.}
\end{figure}

As previously shown, variables are declared with the usage of the \texttt{let} keyword. Besides, after declaring a variable, the user is allowed to assign different values of different types at different points in time to the declared variable. To do that, the assignment operator (\texttt{=}) comes into play. However, before showing examples of code snippets that make use of this operator, it's very important to mention two semantic details of assignment in Bleach:
\begin{itemize}
    \item In Bleach, an assignment is an expression, not a statement. This means that every assignment produces a value.
    \begin{figure}[H]
        \centering
        \begin{lstlisting}
let foo;
print foo = 20; // 20
        \end{lstlisting}
        \caption{Example showing the behavior of an assignment expression.}
    \end{figure}
    
    \item Bleach allows the user to assign a value to more than one variable in an assignment expression.
    \begin{figure}[H]
        \centering
        \begin{lstlisting}
let x = 20;
let y = 42;

x = y = 13;

print x; // 13;
print y; // 13;
        \end{lstlisting}
        \caption{Example showing the possibility of assigning a value to multiple variable in a single assignment expression.}
    \end{figure}
\end{itemize}

Lastly, the Bleach programming language has support for variable shadowing. In practice, this means that when a variable is declared in an inner scope with the same name of another variable, which was declared in an outer scope, the inner variable "shadows" or hides the variable declared in the outer scope. This means that, until this inner scope ends, every variable reference will "hit" the inner one, not the outer one. In this matter, it is important that in order to create a new scope, the user just needs to use curly braces (\texttt{\{\}}). The two examples presented below show how variable shadowing works in practice in Bleach:
\begin{figure}[H]
    \centering
    \begin{lstlisting}
let a = 42;

print a; // 42

{
  let a = "Hello, there!";
  print a; // "Hello, there!" --> In this scope, the variable a that holds the value "Hello, there!" hides the other one that has the same name but stores a different value: 42.
}

print a; // 42
    \end{lstlisting}
    \caption{First example of variable shadowing in Bleach.}
\end{figure}

\begin{figure}[H]
    \centering
    \begin{lstlisting}
let foo = "hi";
print foo; // "hi"

function f(){
  let foo = 42;
  print foo; // 42;

  return;
}
f();

print foo; // "hi"
    \end{lstlisting}
    \caption{Second example of variable shadowing in Bleach.}
\end{figure}


\subsection{Operators}
Operators in a programming language are tools responsible for executing an operation on one or more operands in order to produce a result.

Operators are fundamental building blocks in every programming language, since they allow data manipulation, computations execution, values comparison, and much more.

This subsection is dedicated to introduce the operators Bleach grants to its users and how each of such operators behave. In the end, an operator precedence table will be shown.

\begin{itemize}
    \item \textbf{Unary Operators:} These operators expects just one operand. Bleach has 2 operators that fall in this class.
        \begin{itemize}
            \item \textbf{Arithmetical:}
                \begin{itemize}
                    \item \textbf{Negation} (\texttt{-}): Negates the value of an operand of type \texttt{num}.
                    \begin{figure}[H]
                        \centering
                        \begin{lstlisting}
let n = 42;
print -n; // -42
                        \end{lstlisting}
                        \caption{Example of usage of the "negation" operator (\texttt{-}).}
                    \end{figure}
                \end{itemize}
            \item \textbf{Logical:}
                \begin{itemize}
                    \item \textbf{Not} (\texttt{!}): Inverts the value of an operand of type \texttt{bool}.
                    \begin{figure}[H]
                        \centering
                        \begin{lstlisting}
let b = true;
print !b; // false
print !!b; // true
                        \end{lstlisting}
                        \caption{Example of usage of the "not" operator (\texttt{!}).}
                    \end{figure}                    
                \end{itemize}
        \end{itemize}
    \item \textbf{Binary Operators:} These operators expects just two operands. Bleach has 13 operators that fall in this class.
        \begin{itemize}
            \item \textbf{Arithmetical:}
                \begin{itemize}
                    \item \textbf{Addition} (\texttt{+}): This operator expects two operands, which can be of type \texttt{num} or \texttt{str}. However, the action performed by this operator at runtime depends on the types of the operands provided.

                    If both operands are values of type \texttt{num}, then this operator adds the first (left) operand to the second (right) operand and returns the result (a value of type \texttt{num}).
                    \begin{figure}[H]
                        \centering
                        \begin{lstlisting}
print 2 + 3; // 5
print 2.71 + 3.14159; // 5.85159
                        \end{lstlisting}
                        \caption{First example of usage of the "addition" operator (\texttt{+}).}
                    \end{figure}

                    If both operands are values of type \texttt{str}, then this operator concatenates the first (left) operand to the second (right) operand and returns the result (a value of type \texttt{str}).
                    \begin{figure}[H]
                        \centering
                        \begin{lstlisting}
print "hello," + " there!"; // "hello, there!"
print "a" + "b"; // "ab"
                        \end{lstlisting}
                        \caption{Second example of usage of the "addition" operator (\texttt{+}).}
                    \end{figure}

                    If one operand is a value of type \texttt{num} and the other one is a value of type \texttt{str}, then the one that is a \texttt{num} is converted into its \texttt{str} representation and concatenated to the other operand. The operator then returns the result (a value of type \texttt{str}).
                    \begin{figure}[H]
                        \centering
                        \begin{lstlisting}
print 2 + "two"; // "2two"
                        \end{lstlisting}
                        \caption{Third example of usage of the "addition" operator (\texttt{+}).}
                    \end{figure}
                    
                    
                    \item \textbf{Subtraction} (\texttt{-}): This operator expects two operands of type \texttt{num}. It subtracts the second (right) operand from the first (left) operand, and returns the result (a value of type \texttt{num}).
                    \begin{figure}[H]
                        \centering
                        \begin{lstlisting}
print 5 - 3; // 2
                        \end{lstlisting}
                        \caption{Example of usage of the "subtraction" operator (\texttt{-}).}
                    \end{figure}
                    
                    \item \textbf{Multiplication} (\texttt{*}): This operator expects two operands of type \texttt{num}. It multiplies the first (left) operand by the second (right) operand, and returns the result (a value of type \texttt{num}).
                    \begin{figure}[H]
                        \centering
                        \begin{lstlisting}
print 1.5 * 4; // 6
                        \end{lstlisting}
                        \caption{Example of usage of the "multiplication" operator (\texttt{*}).}
                    \end{figure}
                    
                    \item \textbf{Division} (\texttt{/}): This operator expects two operands of type \texttt{num}. It divides the first (left) operand, also called dividend, by the second (right) operand, also called divisor, and returns the result of the division (a value of type \texttt{num}). It is worth mentioning that if the value of the divisor is \texttt{0}, then a runtime error will be thrown.
                    \begin{figure}[H]
                        \centering
                        \begin{lstlisting}
print 5 / 2; // 2
print 1 / 3; // 0.333333333333333
                        \end{lstlisting}
                        \caption{Example of usage of the "division" operator (\texttt{/}).}
                    \end{figure}

                    
                    \item \textbf{Remainder} (\texttt{\%}): This operator expects two operands of type \texttt{num}. It divides the first (left) operand, also called dividend, by the second (right) operand, also called divisor, and returns the remainder of this division (a value of type \texttt{num}). It is worth mentioning that if the value of the divisor is \texttt{0}, then a runtime error will be thrown. Moreover, a runtime error will also be thrown if both operands are not integer numbers. If the reader wants to computer the remainder of a division between decimal numbers, then he/she/they should use the  \texttt{std::math::fmod} native function, which is presented at section 4.3.9.
                    \begin{figure}[H]
                        \centering
                        \begin{lstlisting}
print 5 % 2; // 1
print 1 % 3; // 1
print -10 % 4 // -2
                        \end{lstlisting}
                        \caption{Example of usage of the "remainder" operator (\texttt{\%}).}
                    \end{figure}
                
                
                \end{itemize}
                
            \item \textbf{Comparison:}
                \begin{itemize}
                    \item \textbf{Greater Than} (\texttt{>}): This operator expects two operands of type \texttt{num}. It checks if the first (left) operand is greater than the second (right) operand. If that is indeed the case, it returns \texttt{true}. Otherwise, it returns \texttt{false}.
                    \begin{figure}[H]
                        \centering
                        \begin{lstlisting}
print 5 > 2; // true
print 1 > 3; // false
                        \end{lstlisting}
                        \caption{Example of usage of the "greater than" operator (\texttt{>}).}
                    \end{figure}

                    \item \textbf{Greater Than or Equal} (\texttt{>=}): This operator expects two operands of type \texttt{num}. It checks if the first (left) operand is greater than or equal to the second (right) operand. If that is indeed the case, it returns \texttt{true}. Otherwise, it returns \texttt{false}.
                    \begin{figure}[H]
                        \centering
                        \begin{lstlisting}
print 5 >= 2; // true
print -1 >= -1 // true
print 1 >= 3; // false
                        \end{lstlisting}
                        \caption{Example of usage of the "greater than or equal" operator (\texttt{>=}).}
                    \end{figure}

                    \item \textbf{Lesser Than} (\texttt{<}): This operator expects two operands of type \texttt{num}. It checks if the first (left) operand is lesser than the second (right) operand. If that is indeed the case, it returns \texttt{true}. Otherwise, it returns \texttt{false}.
                    \begin{figure}[H]
                        \centering
                        \begin{lstlisting}
print 5 < 2; // false
print 1 < 3; // true
                        \end{lstlisting}
                        \caption{Example of usage of the "lesser than" operator (\texttt{<}).}
                    \end{figure}

                    
                    \item \textbf{Lesser Than or Equal} (\texttt{<=}): This operator expects two operands of type num. It checks if the first (left) operand is lesser than or equal to the second (right) operand. If that is indeed the case, it returns \texttt{true}. Otherwise, it returns \texttt{false}.
                    \begin{figure}[H]
                        \centering
                        \begin{lstlisting}
print 5 <= 2; // false
print 0 <= 0; // true
print 1 <= 3; // true
                        \end{lstlisting}
                        \caption{Example of usage of the "lesser than or equal" operator (\texttt{<=}).}
                    \end{figure}
                \end{itemize}
                
            \item \textbf{Equality:}
                \begin{itemize}
                    \item \textbf{Equal} (\texttt{==}): This operator expects two operands of the following built-in types (\texttt{bool}, \texttt{nil}, \texttt{num}, \texttt{str}). It checks whether the values are of the same type and, if that's the case, checks whether such values are the same. Returns \texttt{true} if both conditions are true. Otherwise, returns \texttt{false}.
                    \begin{figure}[H]
                        \centering
                        \begin{lstlisting}
print 2 == 2; // true
print 2 == (1 + 1); // true
print 2 == 3; // false
print "hello" == "hello"; // true
print "hello" == "hell"; // false
print 2 == nil; // false
print nil == nil; // true
print true == true; // true
print true == false; // false
print true == !!true; // true
                        \end{lstlisting}
                        \caption{Example of usage of the "equal" operator (\texttt{==}).}
                    \end{figure}

                    
                    \item \textbf{Not Equal} (\texttt{!=}): This operator expects two operands of the following built-in types (\texttt{bool}, \texttt{nil}, \texttt{num}, \texttt{str}). It checks whether the values are not of the same type and, if they are of the same type, it then checks whether such values are not the same. Returns \texttt{true} if one (or both) conditions mentioned above are not satisfied. Otherwise, returns \texttt{false}.
                    \begin{figure}[H]
                        \centering
                        \begin{lstlisting}
print 2 != 2; // false
print 2 != (1 + 1); // false
print 2 != 3; // true
print "hello" != "hello"; // false
print "hello" != "hell"; // true
print 2 != nil; // true
print nil != nil; // false
print true != true; // false
print true != false; // true
print true != !!true; // false
                        \end{lstlisting}
                        \caption{Example of usage of the "not equal" operator (\texttt{!=}).}
                    \end{figure}
                \end{itemize}
                
            \item \textbf{Logical:}
                \begin{itemize}
                    \item \textbf{And} (\texttt{and}): This operator returns \texttt{true} if, and only if, both operands are "truthy" values. Otherwise, it returns \texttt{false}. This operator performs short-circuiting whenever possible.
                    \begin{figure}[H]
                        \centering
                        \begin{lstlisting}
print 5 and 2; // true
print 5 and false; // false
print false and nil; // false
                        \end{lstlisting}
                        \caption{Example of usage of the "and" operator (\texttt{and}).}
                    \end{figure}
                    
                    \item \textbf{Or} (\texttt{or}): This operator returns \texttt{true} if one of its operands is a "truthy" value. Otherwise, it returns \texttt{false}. As the operator above, this one also performs short-circuiting whenever possible.
                                        \begin{figure}[H]
                        \centering
                        \begin{lstlisting}
print 5 or 2; // true
print 5 or false; // true
print false or nil; // false
                        \end{lstlisting}
                        \caption{Example of usage of the "or" operator (\texttt{or}).}
                    \end{figure}
                \end{itemize}
        \end{itemize}
    \item \textbf{Ternary Operator:}
        A ternary operator is, as its name suggests, an operator that expects three operands. Bleach has just one operator that falls in this class (\texttt{? :}), the one the reader is probably familiar with from C and C++.
    
        This operator provide another way of executing conditional operations. It is used to evaluate a condition and return one of two values based on whether the condition evaluates to \texttt{true} or \texttt{false}.

        The three operands expected by the ternary operator can be of any type (built-in or user-defined). The first one is evaluated by the ternary operator with respect to its truthiness of falseness. If the value is "truthy", then the operator returns the second operand. Otherwise, it returns the third operand.

        \begin{figure}[H]
            \centering
            \begin{lstlisting}
print 2 == 2 ? "2 is equal to 2" : "2 is not equal to 2"; // "2 is equal to 2"
            \end{lstlisting}
            \caption{Example of usage of the "ternary" operator (\texttt{? :}).}
        \end{figure}

    \item \textbf{Bleach's Operators Precedence Table:}
        \begin{table}[H]
        \centering
        \begin{tabular}{|c|c|}
            \hline
            \textbf{Precedence} & \textbf{Operators} \\  % First row: header
            \hline
            1 & \texttt{!}, \texttt{-} (unary) \\  % Start adding data
            \hline
            2 & \texttt{*}, \texttt{/}, \texttt{\%} \\
            \hline
            3 & \texttt{+}, \texttt{-} (binary) \\
            \hline
            4 & \texttt{>}, \texttt{>=}, \texttt{<}, \texttt{<=} \\
            \hline
            5 & \texttt{==}, \texttt{!=} \\
            \hline
            6 & \texttt{and} \\
            \hline
            7 & \texttt{or} \\
            \hline
            8 & \texttt{? :} \\
            \hline
            9 & \texttt{=} (assignment) \\
            \hline
        \end{tabular}
        \caption{Bleach's Operator Precedence Table.}
        \label{tab:example}
        \end{table}
\end{itemize}

\subsection{Control-Flow Structures}
Control-Flow structures are entities that allow the programmer to dictate in what order and by how many times instructions or code blocks are executed.

These entities are responsible for allowing branching mechanisms, decision-making and repetition in a program. The main consequence of this is that a program can have a more dynamic and sophisticated behavior.

In Bleach, there are 3 classes of control-flow structures:
\begin{itemize}
    \item \textbf{Conditional Statements:} These statements are known to execute a block of code based on the fact that certain condition(s) are satisfied. Bleach has support for \texttt{if} statements, which may be followed by any amount of \texttt{elif} statements and a unique \texttt{else} statement, in this order. Bleach also provides support for nesting \texttt{if} statements, which gives more flexibility to the user.
\begin{figure}[H]
    \centering
    \begin{lstlisting}
let number = 15;

if(number % 3 and number % 5){
  print "Fizz Buzz"; // Executes this statement.
}elif(number % 3){
  print "Fizz";
}elif(number % 5){
  print "Buzz";
}else{
  print number;
}
    \end{lstlisting}
    \caption{First example that shows the usage of \texttt{if}, \texttt{elif} and \texttt{else} in Bleach.}
\end{figure}

\begin{figure}[H]
    \centering
    \begin{lstlisting}
let number = 10;

if(number > 0){
  if(number % 2 == 0){
    // Executes this block of code.
    print "The number " + number + " is positive and even!";
  }else{
    print "The number " + number + " is positive and odd!";
  }
}else{
  print "The number " + number + " is negative!";
}
    \end{lstlisting}
    \caption{Second example that shows the usage of \texttt{if}, \texttt{elif} and \texttt{else} in Bleach.}
\end{figure}

    It is important to mention that the \texttt{if}, \texttt{elif} and \texttt{else} statements require a block after it. Otherwise, a syntax error will be thrown by the interpreter. This happens because such statements create a new local scope.
    
    \item \textbf{Loop statements:} These ones are used to allow the repetition of a code block based on a certain condition. Bleach has support for 3 different types of loop statements: \texttt{while}, \texttt{do-while} and \texttt{for}. Just as in conditional statements, Bleach also allows the nesting of loop statements.
\begin{figure}[H]
    \centering
    \begin{lstlisting}
let counter = 10;

while(counter > 0){
  counter = counter - 1;
}

print counter; // 0
    \end{lstlisting}
    \caption{Example that shows the usage of a \texttt{while} loop in Bleach.}
\end{figure}

\begin{figure}[H]
    \centering
    \begin{lstlisting}
let counter = 1;

do{
  counter = counter - 1;
}while(counter > 1);

print counter; // 0
    \end{lstlisting}
    \caption{Example that shows the usage of a \texttt{do-while} loop in Bleach.}
\end{figure}

\begin{figure}[H]
    \centering
    \begin{lstlisting}
let number;

for(number = 10; number > 0; number = number - 1){}

print number; // 0
    \end{lstlisting}
    \caption{Example that shows the usage of a \texttt{for} loop in Bleach.}
\end{figure}

    It is important to mention that these statements also require a block after them. Otherwise, a syntax error will be thrown by the interpreter. This happens because they also create a new local scope, just as the conditional statements.

    \item \textbf{Jump statements:} Jump statements are used by the programmer to alter the execution flow of the source code by "jumping" to a different part of it. Bleach has support for 3 types of jump statements: \texttt{break}, \texttt{continue} and \texttt{return}.

\begin{figure}[H]
    \centering
    \begin{lstlisting}
let numbers = [];

for(let i = 0; i < 10; i = i + 1){
  if(i == 5){
    break;
  }
  numbers.append(i);
}

print numbers; // [0, 1, 2, 3, 4]
    \end{lstlisting}
    \caption{Example that shows the usage of a \texttt{break} statement in Bleach.}
\end{figure}

\begin{figure}[H]
    \centering
    \begin{lstlisting}
let numbers = [];

for(let i = 0; i <= 10; i = i + 1){
  if(i % 2 == 1){
    continue;
  }
  numbers.append(i);
}

print numbers; // [0, 2, 4, 6, 8, 10]
    \end{lstlisting}
    \caption{Example that shows the usage of a \texttt{continue} statement in Bleach.}
\end{figure}

\begin{figure}[H]
    \centering
    \begin{lstlisting}
function foo(){
  print "alpha";

  return;

  print "omega"; // Not reachable due to the "return" keyword.
}

foo(); // "alpha"
    \end{lstlisting}
    \caption{Example that shows the usage of a \texttt{return} statement in Bleach.}
\end{figure}
\end{itemize}

\subsection{Functions}
\textbf{Falar como ocorre a passagem de valores em chamadas de função: por valor ou por referência.}

Functions, as is widely known, are fundamental building blocks in programming. They are reusable blocks of code created with the intention of executing a specific job. Therefore, they allow programmers to organize, reuse, and manage code in a more organized manner.

Functions can take inputs, process them and also return a value. As mentioned above, they are very useful because they are helpful when it comes down to breaking complex problems into smaller and simpler chores.

The code snippet below shows an example of a function declaration statement in Bleach:

\begin{figure}[H]
    \centering
    \begin{lstlisting}
function functionName(parameter1, parameter2, parameter3){
  // Code to be executed
  return value;
}
    \end{lstlisting}
    \caption{Example of function declaration statement in Bleach.}
\end{figure}

Still regarding function declaration statements, it is relevant to mention that if the programmer omits the \texttt{return} statement from it, then when such function is called at runtime it will, by default, return the \texttt{nil} value.

In Bleach, function calls are no different than in the most famous programming languages. The programmer is expected to call it by its name and provide the necessary arguments, according to its respective declaration statement. The code snippet below shows how to properly call a function in Bleach.

\begin{figure}[H]
    \centering
    \begin{lstlisting}
function subtract(a, b){
  return a + b;
}

let x = 5;
let y = 3;
let answer = subtract(a, b);

print answer; // 2
    \end{lstlisting}
    \caption{Example of a function call in Bleach.}
\end{figure}

As happens in most dynamically-typed programming languages, when a function call does not comply with its respective declaration statement, a runtime error is thrown.

It is very important to mention that functions in Bleach have certain limitations: They don't have support optional parameters, nor for default value for its parameters, and also cannot be overloaded. \textbf{("Future Work" Material? Assignment to students?)}

Going further, in Bleach, functions are first-class values. Basically, this means that a function is not different from any other value in Bleach. This fact implies in the consequences listed below:

\begin{itemize}
    \item Functions can be assigned to variables.
    \item Functions can be passed as arguments to other functions.
    \item Functions can be returned from other functions.
    \item Functions can be stored inside data structures.
\end{itemize}

Finally, Bleach also has support for anonymous functions, also known as lambda functions. As the name suggests, these are functions that do not have a name attached to them (i.e., they are declared without a name). Usually, famous programming languages have syntax mechanisms that permit such a thing. Bleach opted for taking a mixed approach by using the \texttt{lambda} keyword, used in Python, as well as an "arrow" token (\texttt{->}). When it comes to semantics, anonymous functions are just as powerful as the default ones.

The code snippet shown below demonstrates how to use an anonymous function in Bleach:

\begin{figure}[H]
    \centering
    \begin{lstlisting}
    let add = lambda -> (a, b){ return a + b; }
    
    let answer = add(10, 15);
    
    print answer; // 25
    \end{lstlisting}
    \caption{Example of an anonymous function usage in Bleach.}
\end{figure}


\subsection{The Object-Oriented Paradigm in Bleach}
\textbf{(Não sei se essa subseção ficou muito bem escrita. Poderia dar uma olhada com mais atenção nela?)}

As previously stated, Bleach is a multi-paradigm programming language, since it has support both the procedural and the object-oriented paradigms.

When it comes to the object-oriented paradigm, Bleach has support to the following features of it, such as: classes, attributes/fields, methods, instances, the \texttt{self} keyword and inheritance. These features allow the reader to build modular, reusable, and maintainable code by organizing software into objects, entities that are capable of representing both data and behavior.

The rest of this subsection is dedicated to provide a more in-depth overview about each of the mentioned features:

\begin{itemize}
    \item \textbf{Classes:} As is common knowledge among undergraduate CS students, in the object-oriented paradigm, a class is basically a "blueprint" for creating instances/object. Regarding this, since Bleach is a dynamically-typed language, it made sense for it to take the same route as Python: It does not require explicit type information about attributes and methods. Furthermore, there is a lot of flexibility when it comes to additions, because the programmer can add new attributes and methods at runtime to instances of a class, as will be shown later in this subsection. In order to declare a new class, the user must use the \texttt{class} keyword followed by the name of the declared class and curly braces (\texttt{\{\}}):
\begin{figure}[H]
    \centering
    \begin{lstlisting}
// The declaration of a class called "Person".
class Person{
  method init(name, age){
    self.name = name;
    self.age = age;
  }

  method greet(){
    return "I am " + self.name + " and I have " + self.age + " years.";
  }
}
    \end{lstlisting}
    \caption{Example of a class declaration in Bleach.}
\end{figure}

    \item \textbf{Instances/Objects:} When it comes to instances/objects, Bleach took the same approach that was taken by Lox and Python. An instance/object is a concretization of a class. In other words, an instance is an entity created at runtime that has the structure and behavior defined by its class declaration. However, it is important to remember that each instance of a class can store its own unique data. In essence, an instance of a class has its own copy of the class’s attributes, but they share the class's methods.

    Since Bleach took the same approach to deal with instances/objects as Lox, it is safe to say that the following quote written by Robert Nystrom in \cite{nystrom2021crafting} also applies to Bleach: "Instances are loose bags of data and you can freely add fields to them as you see fit using normal imperative code."
    
    \item \textbf{The \texttt{self} keyword:} The keyword \texttt{self} serves for a very specific purpose inside a class: It's a reference to the current instance/object of the class. This keyword allows methods within a class to access the specific instance's attributes and other methods.
    
    In addition to that, the \texttt{self} keyword assists to clarify when local variables (parameters that receive passed arguments in a method call, for example) have the same name as instance variables.

    Finally, this keyword permits methods to return the current instance, which allows method chaining: a scenario where multiple methods can be called in sequence on the same instance.

    The code snippet shown below demonstrates a more sophisticated use of the \texttt{self} keyword:
\begin{figure}[H]
    \centering
    \begin{lstlisting}
class Counter{
  method init(){
    self.count = 0;
  }

  method increment(){
    self.count = self.count + 1;
    return self;
  }
}
let counter = Counter();
counter.increment().increment().increment();
print counter.count; // 3;
    \end{lstlisting}
    \caption{Example of a more refined use of the \texttt{self} keyword in Bleach.}
\end{figure}
    
    \item \textbf{Attributes:} As seen in the item above, attributes don't need to have type annotations attached to them. Instead, the attributes' types are determined when values are assigned to them at runtime. On top of that, attributes of a class can store values of different types at different points in time. As in most programming languages that are object-oriented, in order to access the attributes of a class, one must use the "dot" (\texttt{.}) notation, as displayed below:
\begin{figure}[H]
    \centering
    \begin{lstlisting}
class Square{
  method init(length){
    self.length = length;
  }
}

let square = Square(5);

print square.length; // 5

square.area = square.length * square.length;
print square.area; // 25

square.perimeter = 4 * square.length;
print square.perimeter; // 20
    \end{lstlisting}
    \caption{Example that illustrates the usage of attributes of a class in Bleach.}
\end{figure}

    \textbf{Falar sobre a implicação semântica de ter um método e um atributo com o mesmo nome em uma instância de uma classe.}

    \item \textbf{Methods:} In Bleach, methods of a class don't have type annotations in their signature, just like functions. The types are checked during runtime when the method is called, which can potentially throw a runtime error if a value of an unexpected type is passed as an argument. In order to declare a method of a class in Bleach, the programmer can use the same syntax as in a function declaration, just needing to replace the \texttt{function} keyword by the \texttt{method} keyword. To call a method of a class, it is required to use the "dot" notation (\texttt{.}) combined with the function calling convention (\texttt{()}), as exhibited below:
\begin{figure}[H]
    \centering
    \begin{lstlisting}
class Dog{
  method bark(){
    return "Owf Owf!";
  }
}

let dog = Dog();

print dog.bark(); // "Owf Owf!"
    \end{lstlisting}
    \caption{Example that illustrates the usage of methods of a class in Bleach.}
\end{figure}

    Last but not least, since default functions and anonymous functions are values in Bleach, it is completely possible to add new methods to an instance of a class by using them, as demonstrated below:
\begin{figure}[H]
    \centering
    \begin{lstlisting}
class Dog{
  method bark(){
    return "Owf Owf!";
  }
}

let dog = Dog();

dog.louderBark = lambda -> (){ return "WOOF!"; };
print dog.louderBark();
    \end{lstlisting}
    \caption{Example that illustrates the addition of a new method to an instance of a class during runtime.}
\end{figure}
    
    \textbf{Falar sobre os métodos especiais "init" e "str". Ambos podem ser omitidos na declaração de uma classe. O que acontece quando eles são omitidos? Quais as implicações disso?}

    
    \item \textbf{Inheritance and the \texttt{super} keyword:} In Bleach, as in other popular languages like C++, Java and Python, inheritance allows the user to create a new class based on an existing class, extending or modifying its functionality.

    Bleach follows a very similar implementation of inheritance to that of Python with just one major difference: For simplicity purposes, Bleach only supports single inheritance whereas Python has support for multiple inheritance.

    In order to create a class that inherits from another, the use of the \texttt{inherits} keyword is required as it will be shown below.

    Lastly, taking inspiration from Python, Bleach also has support for the \texttt{super} keyword. This one is used in order to refer to the methods of a superclass (i.e., the base class) from within a subclass (i.e., the derived class). To put it in another way, the \texttt{super} keyword allows a subclass invoke methods that were defined in its superclass (if any), even if such methods were overridden inside the subclass. 

    The code snippet below shows how to properly use inheritance in Bleach:
\begin{figure}[H]
    \centering
    \begin{lstlisting}
class Animal{
  method init(name){
    self.name = name;
  }
  method str(){
    return self.name + " makes a sound.";
  }
}
class Dog inherits Animal{
  method init(name, breed){
    super.init(name);
    self.breed = breed;
  }
  method str(){
    return self.name + " is a " + self.breed + " and barks.";
  }
}

let dog = Dog("Thor", "Rottweiller");
print dog; // "Thor is a Rottweiller and barks."
    \end{lstlisting}
    \caption{Example that shows the usage of inheritance in Bleach.}
\end{figure}


    
\end{itemize}


\subsection{Bleach Native Functions}

\section{Bleach Interpreter Components Breakdown}

\subsection{Bleach Tree-Walk Interpreter Overview}
\subsection{Lexer/Scanner}
\subsection{Parser}
\subsection{Resolver}
\subsection{Interpreter/Runtime}

\section{Challenges, Decisions and Trade-Offs}
\textbf{Falar sobre a decisão de permitir a redeclaração de variáveis globais em Bleach e o porquê disso.}

\textbf{Falar sobre a decisão de implementar single-inheritance ao invés de multiple-inheritance. Isso pode entrar tanto como "Future Work" como ser um assignment para uma turma.}

\textbf{Falar sobre a decisão de usar um Handwritten Lexer/Scanner.}

\textbf{Falar sobre a decisão de usar Recursive-Descent Parsing.}

\textbf{Falar sobre a decisão de usar um Tree-Walk Interpreter ao invés de uma ByteCode VM ou de um Compilador.}


\section{What Makes Bleach Shine and How It Can Be Used In a Classroom Environment}